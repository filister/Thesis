\chapter{Literature Review}
\section{Mathematical Models}

Mathematical models are a useful tool that explains the dynamics of a disease in population. Even the simplest of models does provide useful insights vital for understanding complex processes \cite{brauer2001mathematical}. A mathematical models allows for conceptual experiments that would be physically difficult or impossible to do \cite{aron2007mathematical}. We can extrapolate from available information to be able to predict how a disease outbreak will evolve. The ability to predict ensures that appropriate measures of curbing the spread of a disease through the whole population are put in place. Mathematical models are also a very useful tool for assessing the impact of these control measures \cite{keeling2009mathematical}.

\section{Models With Age Structure}
Structured population models distinguish individuals from one another according to characteristics such as age, size, location, status, and movement, to determine the birth, growth and death rates, interaction with each other and with environment, infectivity \cite{auger2008structured}



The age structured models capture the effects of demographic behaviour of individuals \cite{liu2015stability}. According to \cite{alexanderian2011age} age structured models are the most appropriate in the study of cholera since both vaccine efficacy as well as the risk of contracting cholera depends on the age of an individual. \cite{li2008continuous} emphasised on the fact that even the simplest models do cater for  some separate groupings in the population by dividing the population into different disease categories. They also highlighted that age may have some influence on reproduction, survival rates and behaviours. Here they noted that behavioural changes are very crucial as they are the major focus in the control and prevention of many infectious diseases. Let us also note that behavioural  changes are  an important aspect in the control and prevention of substance abuse.

Age structured models are used in studying various diseases and conditions. \cite{shim2006age} used to model the transmission of rotaries infection in the presence of maternal antibodies and vaccination. They made the  assumption that contacts between individuals are influenced by age-class activity levels as well age densities. Their population is divided into 6 age dependent classes where the age dependent mixing contacts structure is modelled via the mixing density $p(t,a,a')$ which gives the proportion of contacts between individuals of age $a$ and those aged $a'$ supposing that they have had some contact at time $t$. The basic reproduction and the control adjusted number are computed from the Lotka characteristic equation.


\cite{castillo2002age} discusses an age structured core group model and its impact on STD dynamics. An assumption of proportionate mixing is made and also assumes that the population has reached its stable distribution.In their work they apply the theorem that checks the two conditions, namely that $R_0 <1$ and $B$ is uniformly Lipshitz continuous on $\mathbb{R}^+$ to show that the disease free equilibrium is locally asymptotically stable. This is proved by applying the semi-group approach.

\section{Models Of Substance Abuse}

Epidemiological models are used to study the dynamics involved in the initiation and use of drugs because substance abuse is naturally contagious \cite{agestructureddrug}. In the case of infectious diseases there is need for an agent that is transmitted via physical contact while substance abuse is spread as an innovative socially acceptable practice to those susceptible to substance  abuse \cite{kalula2012}. Studying the evolution of substance abuse is more complex because we must not only consider the susceptible individual's immediate contacts as the only forces behind possible initiation. Another very influential force can be the overall perception of drugs in the susceptible person's society as sometimes portrayed in movies and news \cite{agestructureddrug}.

A common assumption in many models of substance abuse is that before a susceptible individual progress into the hard drug use state they are first recruited as light drug users(\cite{agestructureddrug},\cite{kalula2012},\cite{nyabadza2010}). The supporting assumption is also that light drug users provide a positive feedback that has more potential of initiating susceptible persons into substance abusing. On the other hand hard drug users are perceived negatively and alert people to the danger of substance abuse (\cite{agestructureddrug},\cite{kalula2012},\cite{nyabadza2010}).

Mathematical models have been formulated and explored in the province of Western Cape recently by ((\cite{kalula2012},\cite{nyabadza2010})). Other recent models of substances of substances have been formulated for other substances such as alcohol \cite{alcohol2007}.

Kalula and Nyabadza  considered a model for stimulants such as methamphetamine as the substance of abuse. This model divides the total population under consideration into core and non core groups . Six states are considered for the model where five states compartmentalise the population in the non core group.The model is shown to be well posed by establishing a feasible region that is positively invariant where the state variables remained non negative for positive initial  conditions. $R_0$ is defined to represent the average number of secondary cases that one drug user can generate in a population of potential drug users. $R_0$ is calculated using the next generation method. Numerical simulations to verify the theorem on stability of the drug free equilibrium are carried out using the Runge Kutta scheme in Matlab. A critical value of $R_0$ is established below which no drug persistent equilibria exists. It is not enough for $R_0 <1$ but for an effective drug abuse control the reproduction number must be brought below the critical reproduction number. The role of key parameters on the value of the reproduction number was investigated and it was established that reducing $R_0$ can be achieved by accelerating the rate of transference into the hard drug use compartment and the rate into the quitters compartment and into the rehabilitation compartment. The model proposes that in order to effectively fight the substance abusing problem it is imperative to limit the time spend in the light substance abusing compartment since these individuals are assumed to be more capable of recruiting more people than hard drug users. The data applied for the model was that of methamphetamine users in South Africa that sought treatment for the ten year period up to 2009 \cite{kalula2012}.

Nyabadza and Hove Musekwa formulated a model studying the dynamics of amphetamine use in the Western Cape Province. The aim of the model is to enable the prediction of the prevalence of drug use. Four compartments are included in the model. Data that was used for fitting was from the period of 1996 to 2008 which is the data for amphetamine abusers in treatment. An assumption of the model is that those who quit while in treatment move to the compartment of permanent quitters who can relapse after they have first recovered. Another very realistic assumption is that a quitter who relapses will most likely move straight into the hard drug user state because of the previous familiarity with abusing drugs. The incidence function includes an exponential function so as to cater for behaviour change in the model. Behaviour change is likely to result if a susceptible individual is exposed to the adverse effects of drugs such as death as they occur to a person already abusing substances. The model reproduction number is defined as the number of new initiates generated by one index case in a population that is entirely susceptible. It is a threshold number that determines the persistence of amphetamine abuse.
Reduction in reproduction number results in the reduction in the number of drug users. Practical ways as informed by the model is to reduce the contact rate between the susceptible and those on drugs, encourage increase in the behaviour change as well as reduce the time that those on drugs spend in the light drug use state where they have the greatest potential to recruit more susceptible people \cite{nyabada2010}.


The model considered by Fabio Sanchez et al  is a classic SIR epideomological framework where the population is divided into the compartments occasional and moderate drinkers (S) , problem drinkers (D) and temporarily recovered (R). Uptake into the susceptible class is considered as the number of new recruits that join the population as moderate and occasional drinkers. In the analysis of this model the drinking free equilibrium is defined as the state where a drinking culture does not exist. The reproduction number is thus  defined as the measure of resilience of the drinking free equilibrium to the invasion of problem drinkers. It gives the ratio of the average number of secondary cases generated by a typical drinker. Two forms of reproduction numbers are defined. A reproduction number $R_0 = \frac{\beta}{\mu}$ is defined in the absence of treatment and there is no class of those recovered from drug use yet. The other formulation of the reproduction number $R_{\phi}=\frac{\beta}{\mu + \phi}$ caters for the existence of recovered people after undergoing treatment. Clearly $R_0$ is greater than $R_{\phi}$ whenever $\phi$ is greater than 0. The implications of these  different reproduction numbers is that $R_0<1$ guarantees that the culture of drinking will not be established as long as the initial number of drinkers is low . On the other hand $R_0>1$ ensures that just the introduction of a single problem drinker will result in the eruption of a culture of drinking. $R_{\phi}<1$ does not guarantee that a culture of drinking will not be established because the $R$ class can produce more people who are  susceptible to alcohol abuse. The possibility of relapsing after recovering makes it imperative for treatment to be effective to avoid the blowing out of the alcohol abuse even with high rates of uptake into rehabilitation and quitting after treatment \cite{alcohol2007}.

Almeder et al formulated a model where the total population is divided into two groups mainly non- users and users. The model is simplified by disregarding the death and migration factors. They also assumed a constant birth cohort size. Movement is only one directional from the non user group to the user group.  User population does not just include the current users but includes everyone who has ever consumed or partaken of drugs.
The model describing the dynamics of non-users is given as a form of the McKendrick equation. This is given as 
\begin{eqnarray}
P_a + P_t = -\mu(a,t) P(a,t)
\end{eqnarray}
where in this case $\mu(a,t)$ represents the initiation rate whereas in our model this represents  the naturally mortality rate. 
This initiation rate $\mu(a,t)$ is assumed to be the product of three different factors namely a basic age specific initiation rate, the influence of the reputation of the drug as well as a prevention factor which includes the effects of age specific prevention programs \cite{agestructureddrug}.












