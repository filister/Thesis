\chapter{Literature Review}

The age structured models capture the effects of demographic behaviour of individuals \cite{liu2015stability}. According to \cite{alexanderian2011age} age structured models are the most appropriate in the study of cholera since both vaccine efficacy as well as the risk of contracting cholera depends on the age of an individual. \cite{li2008continuous} emphasised on the fact that even the simplest models do cater for  some separate groupings in the population by dividing the population into different disease categories. They also highlighted that age may have some influence on reproduction, survival rates and behaviours. Here they noted that behavioural changes are very crucial as they are the major focus in the control and prevention of many infectious diseases. Let us also note that behavioural  changes are  an important aspect in the control and prevention of substance abuse.

Age structured models are used in studying various diseases and conditions. \cite{shim2006age} used to model the transmission of rotaries infection in the presence of maternal antibodies and vaccination. They made the  assumption that contacts between individuals are influenced by age-class activity levels as well age densities. Their population is divided into 6 age dependent classes where the age dependent mixing contacts structure is modelled via the mixing density $p(t,a,a')$ which gives the proportion of contacts between individuals of age $a$ and those aged $a'$ supposing that they have had some contact at time $t$. The basic reproduction and the control adjusted number are computed from the Lotka characteristic equation.


\cite{castillo2002age} discusses an age structured core group model and its impact on STD dynamics. An assumption of proportionate mixing is made and also assumes that the population has reached its stable distribution.In their work they apply the theorem that checks the two conditions, namely that $R_0 <1$ and $B$ is uniformly Lipshitz continuous on $\mathbb{R}^+$ to show that the disease free equilibrium is locally asymptotically stable. This is proved by applying the semi-group approach.
