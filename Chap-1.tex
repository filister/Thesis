\chapter{Introduction}



%%%%%%%%%%%%%%%%%%%%%%%%%%%%%%%%%%%%%%%%%%%%%%%%%%%%%%%%%%%%%%%%%%%%%%%
\section{Substance Abuse}
 
The substance abusing problem is complex and has widespread consequences. Various research findings have consistently pointed to the enormity of the problem worldwide. Research findings estimate that around $246$ million individuals were reported to have used an illicit drug in 2013, with $27$ million  reported to have progressed to become high risk drug users \cite{unodc2015}. High risk drug users is a new term referring to the group of people previously considered as the problem drug users. High risk drug use is defined by the American Monitoring Centre for Drugs and Drug Addiction (EMCDDA) as injecting drug use or long duration or regular use of opioids, cocaine and/or amphetamines \cite{EMCDA}.

The number one drug for which people seek treatment in Africa is cannabis. Substance abuse  in Africa is mainly fuelled  by the continent's role in illicit drug trafficking. Due to various factors the African continent is a transit route for drugs transported across the globe with South Africa considered the regional hub \cite{unodc2015}. Although the drugs into Africa are destined for Europe and North America the transit countries have a habit of becoming user countries \cite{segell1998stability}. As a result African countries are vulnerable to drug abuse along with crime related to drugs. Usually there is increase in organized crime thus further widening the  economic influence of traffickers threatening the security,health and development of the continent.
 
 Alcohol is reported as the most abused substance in South Africa \cite{parry1998substance}. The Western Cape province was reported to have the second highest prevalence of harmful drinking amongst expecting mothers \cite{harker2008substance} . High numbers of Foetal Alcohol Spectrum Disorders (FASD) signify the extend of alcohol abuse among women in the province. The rate of babies born with (FASD) in the province is reported to be among the highest in the country.Substance abuse is defined as the hazardous or harmful use of alcohol and  illicit drugs. Usually  people do not regard alcohol abuse as a real problem for which they should seek treatment. The Ministry of Social Development is mainly responsible for programmes that seek to address substance abuse issues in the Western Cape province. In order to accomplish the task it is imperative that at risk individuals are timeously identified in order for them to be given necessary help before they progress into dependence. 


Drug abuse in South Africa  is mainly driven by increased supply of drugs. Illicit drugs are easily available since the country is the largest transit zone in Africa. Poverty also drives the increase in street level dealing of illicit drugs. Other factors responsible for substance abuse amongst the youth population are family dysfunction and the phenomenon of absent parents. As the drug problem escalates the province is faced with even higher rates of gangsterism. Gangs have extended their turfs into schools where learners are being manipulated as mediums facilitating the sale of drugs in school premises. This has made schools very unsafe as they are characterised by gang violence and robbery.In the Western Cape province alone results from a survey of 133 schools revealed that  $61.6 \%$ of these schools had experienced gang related disturbances with $2$ out of every $5$ schools confirming the presence of drug merchants and peddlers within their schools \cite{socialdevelopwc}.

 In some communities drug dealing is so intricately interwoven into the community micro economics. This usually breeds powerful gang structures with leadership in place  exercising power through the patronage system and they become the source of employment especially in impoverished areas mostly characterised by high levels of unemployment. As these communities heavily depend on the income that comes from drug peddling this problem becomes so entrenched into the community and addressing the supply of drugs in such cases is very challenging and the resulting criminal network are extremely difficult to break\cite{modernisation}.
 
 Cape Town experiences high levels of substance abuse with most people reporting methamphetamine as their primary substance.  There has been an increase in demand of drug abusing treatment \cite{kalula2012theoretical} with methamphetamine accounting for $35 \%$ of patient admission . Other common drugs abused are alcohol and cannabis. The abuse of stimulants such as methamphetamine in Cape Town is also linked to other health related problems since they lead to a  rise in risky sexual behaviour \cite{harker2008substance} a as well as the escalation of social problems associated with abusing substances. 
 
 Various factors influence patterns in substance abuse. Age, social class, occupation, school status ,gender and geographical location are pivotal in establishing the extend of susceptibility of an individual to substance abuse. Peer pressure is what gets most young people involved in drugs amongst younger people. The culture of communal drinking promotes alcohol abuse among adults \cite{parry1998substance}.More factors to consider are chemical dependence on alcohol, poor social conditions and boredom as well as  a lack of social mechanisms that are pt in place to deal with people abusing alcohol and illicit drugs. Reasons why some people get involved in substance abusing activities range from habit, the need to alter mood states, a mechanism coping with stressful situations as well as a way of enjoyment.



Drug abuse is a chronic health condition which is difficult to control because in most places it is consider a criminal offence \cite{unodc2015}. Epidemiological indicators of substance abuse are usually estimated from incomplete data. This hinders efforts to monitor and control the spread of substance abuse  \cite{rossi2003role}. Information on substance abuse is obtained from general population and school surveys, estimates of problem drug use, data collected from treatment centres, information in relation to drug related deaths and also drug related infections such as HIV.

\section{Substances Abused In Cape Town}

\subsection{Methamphetamine}
$98\%$ of Tik addicts who seek help in South Africa are from the Western Cape \cite{mordenisation}. Methamphetamine is a powerful highly addictive stimulant which affects the central nervous system. The street name of the drug in Cape Town is called Tik. Worldwide methamphetamine goes by the name meth, chalk, ice and crystal. It is a white,  colorless, bitter-tasting crystalline powder that dissolves easily in water or alcohol. It is a drug that is medically used for patients who suffers from attention deficit hyperactivity disorder. It is also used as a short component of weight loss treatment . To the user the drug results in a pleasurable sense of well being or euphoria. Greater amounts of the drug gets into the brain and will cause longer lasting harmful effects to the central nervous system. This drug has a high potential for widespread abuse \cite{drugabuse1}.

\subsection{Alcohol}

Alcohol Use Disorder (AUD) is a medical diagnosis for problematic alcohol drinking that becomes severe \cite{nia}. Alcohol addiction is defined distinguished from alcohol abuse with addiction defined to be the psychological and physical dependence on alcohol while alcohol abusers are usually heavy drinkers, not necessarily addicted, who will perpetuate their drinking despite the consequences \cite{drug}.
Abusing alcohol has effects on the functioning of the body thus affecting the mood and behaviour of a person. Usually a person has difficulties thinking and making movement co-ordinations \cite{nia1}. Alcohol consumption is more common among farm workers around South Africa which is a result of the 'dop system' of the apartheid era. This refers to an arrangement where the employee is given alcohol as the benefit for employment \cite{london1999thedop}.  


\subsection{Cannabis}
Cannabis is commonly known as marijuana \cite{hall1998adverse}.Recreational users of cannabis perceive it as harmless. It is obtained from the plant Cannabis sativa and its subspecies. Cannabis contains $\Delta^9$ tetrahydrocannabinol (THC) responsible for marijuana intoxication resulting in its use for recreational purposes \cite{ashton1999adverse}.It can be smoked in hand rolled ,pipes, water pipes,smoked in blunts as well as mixing marijuana in food and ingesting it \cite{nidamarijuana}. The use of cannabis can result in mood altering effects such as euphoria. This ability to produce a high often results in wide spread and often chronic recreational use. Fatuous laughter and talkativeness often results if the substance is taken in a social gathering setting. In naive users the  most common side effects are anxiety, panic reactions, increased risk of accident as well as increased risk of psychotic symptoms (\cite{hall1998adverse},\cite{ashton1999adverse}). The long term use of cannabis increases the risk of respiratory cancer as well as acute and chronic bronchitis. Smoking in pregnancy is indicated for increased risk in birth defects such as ventricular septal defect and low birth weight \cite{csam}.

\section{Project Motivation And Objectives Of The Study}
\subsubsection{Motivation}
 Africa is no longer just the transit route for illicit drugs, but has also become the consumer of these drugs. This has made the problem more complex and leaves in its trail devastating effects \cite{au}. In some suburbs of Cape Town reports of violence and crime related to drug trafficking activities have increasingly become common. There is also a big challenge of providing support for the ever increasing number of drug users a direct result of drug trafficking activities in the city. One of the aims of the African Union as reported in their "Action plan on drug control (2013-2017)" is to increase monitoring of changing and emerging trends of drug use  as well as the implementation of evidence based responses \cite{au1}. To this end it is imperative that in order to gain understanding of the changing and emerging trends we need to consider important  characteristics that are influential in susceptibility to substance abuse. Age is one such factor that needs to be monitored and help detect the changing trends in relation to the substance abusing problem. To this end we seek to formulate an age related model of substance abuse.
\subsubsection{Objectives Of The Study} 
The main objective is to formulate a model of substance abuse incorporating age structure.
\\
Other objectives are:

\begin{itemize}
\item Determine if the model formulated is well posed.\\
\item Compute the basic reproduction number of the model formulated.\\
\item Choose a suitable numerical scheme for approximating the solution and determine its convergence.\\
\item Carry out numerical simulations of the age structured model of substance abuse.\\
\item Fit the model to the Cape Town Data for substance abusing people in rehabilitation.\\
\item To predict the trend of the age related substance abuse for the Cape Town population.\\
\item 
\end{itemize}

 
\section{Mathematical Preliminaries}
\subsection{The McKendrick-von Forster Partial Differential Equation}
The McKendrick equation gives a description of the time evolution of a population that is structured in age. It is one of the ways of modelling the evolution of an age structured population and it takes the form of the following partial differential equation:
\begin{equation}\label{McKendrickog}
\frac{\partial P(a,t)}{\partial t}+ \frac{\partial P(a,t)}{\partial a} = -\mu P(a,t)
\end{equation}

A standard way of solving equation (\ref{McKendrickog})is by using a method of characteristics. This method recognises the existence of curves 
\subsubsection{Method Of Characteristics}
There are curves in the a-t plane called characteristic curves. Along these curves the solution is constant and equal to its initial value.

In order to find the characteristic curves we introduce the following parametric curves
\[a=a(s) \quad \textsl{and}  \quad t=t(s)\]
 and we have the following
 \[K=P(a(s),t(s))\] taking 
 Thus we have 
 \begin{equation}
 \frac{dK}{ds}=\frac{dP(a(s),t(s))}{ds}=\underbrace{\frac{\partial P}{dt} \frac{dt}{ds} + \frac{\partial P}{ds}}_{\text{using the chain rule}}
 \end{equation}
 
 If we chose
 \begin{equation}\label{para1}
\frac{ da}{ds}=1 
 \end{equation}
 and 
 \begin{equation}\label{para2}
 \frac{dt}{ds}=1
 \end{equation}
 we have 
 \begin{equation}
 \frac{dK}{ds}=\frac{\partial P}{dt} +\frac{\partial P }{da}
 \end{equation}
 Eventually we get
 \begin{equation}\label{para3}
 \frac{dK}{ds}= -\mu(a(s)) K
 \end{equation}
 integrating equations(\ref{para1})and (\ref{para2}) we obtain the following characteristic curves
 \[t = t_0 +s \quad \text{and} \quad a=a_0 + s\]
 
 Integrating equation (\ref{para3}) we get 
 \begin{eqnarray}
  \int^{s}_{0} \frac{dK(\alpha)}{K(\alpha)}&=& -\int^{s}_{0} \mu (a(\alpha)) d \alpha \nonumber \\
  \text{ln} K(\alpha)]_{o}^{s}&=& -\int^{s}_{0}\mu (a_0 + \alpha)) d \alpha \nonumber \\
  \text{ln} [\frac{K(s)}{K(0)}] &=& -\int^{a_0 + s}_{a_0} \mu (\theta) d \theta \nonumber \\
  K(s) &=& K_0 e^{- \int^{a_0 + s}_{a_0} \mu (\theta) d \theta} \nonumber \\
  K(s) &=& K_0 \frac{e^{- \int^{a_0 + s}_{0} \mu (\theta) d \theta}}{e^{- \int^{a_0 }_{0} \mu (\theta) d \theta} }
\end{eqnarray}  
  
If we consider the initial condition $P(a,0)$
where $K(s)= P(a(s),t(s))$

Equating $P(a,0)= P(a(s),t(s))$  we get 
\[a(s)=a_0 + s= a\]
\[t(s)=t_0 +s =0\]

Taking \[t_0=0\Longrightarrow t=s \quad \text{and} \quad a_0=a-t\]
But we know that \[K(s)=P(a_0+s,s)\] and \[K(0)=P(a_0,0)=P(a-t,0)\]
Hence \begin{equation}
P(a,t)=P(a-t,0)e^{\int^{a}_{a-t} \mu (\theta) d \theta} \quad \text{for} \quad a >t
\end{equation}

If we consider the initial condition $P(0,t)$
where $K(s)= P(a(s),t(s))$

Equating $P(0,t)= P(a(s),t(s))$  we get 
\[a(s)=a_0 + s= 0\]
\[t(s)=t_0 +s =t\]

Taking \[a_0=0\Longrightarrow a=s \quad \text{and} \quad t_0=t-a\]
But we know that \[K(s)=P(s,t_0 + s)\] and \[K(0)=P(0,t-a)=P(0,t-a)\]
Hence \begin{equation}
P(a,t)=P(0,t-a)e^{\int^{a}_{0} \mu (\theta) d \theta} \quad \text{for} \quad t >a
\end{equation}
 \subsection{The Reproduction Number}
 The reproduction number denoted $R_0$ is defined as the average number of secondary cases generated by a primary case \cite{Chowell155}. This is an important measure in epidemiology that is used in assessing if an infection has the potential to spread in a population.
 
To obtain an expression for the basic reproduction number we make use of the Euler Lotka characteristic equation.
 
\subsubsection{The Euler Lotka Characteristic Equation}
 The Euler Lotka characteristic equation is given as 
 \begin{equation}\label{euler}
 \int_{\alpha}^{\beta} e^{-ra} m(a) l(a) da =1
 \end{equation}

where $l(a)$ is defined as the survival rate and $m(a)$ is defined as the maternity function depicting the birth rate per capita for mothers of age $a$.

Equation (\ref{euler}) has a unique solution $r_0$ which denotes the intrinsic rate of natural increase of a population. This is computed as the dominant real root of the Euler Lotka characteristic equation.

\subsection{Existence And Uniqueness Of Solutions}
One of the issues when dealing with a model that is given as a coupled set of partial differential equations is that it is not easy to derive an analytical solution to the system. As a result we appeal to a numerical solution that best approximates the solution.
Before resorting to a numerical implementation there are some issues that must first be verified. We need to determine if our system of equations is well posed. The main question to be ascertained is that do there exist a solution to our system? If that is satisfied we want the solution to be unique and to continuously depend upon the given initial data.
In order to establish the above conditions we formulate our system as an abstract Cauchy problem of the form

\begin{equation}\label{abstractcauchy1}
\frac{dx}{dt} = A x(t) + f(t,x) 
\end{equation}
The solution of the above equation can be established using the semi group approach as follows
\subsubsection{The Semi Group Approach}
If we consider the following differential equation
\begin{equation}\label{abstractcauchy2}
\frac{dx}{dt} = A x(t), \quad t \geq 0 , \quad x(0)=f \quad \text{on a Banach space X}
\end{equation}
 where $A$ is a given linear operator with domain $D(A)$ and $f \in X $.
 We obtain the generator $A$ of $T(.)$ which is given by setting 
 \[D(A)=\lbrace f\vert \text{lim}_{t\rightarrow 0^{+}} \frac{1}{t}(T(t)f-f) \quad \text{exists}\rbrace \] and is such that
 \[Af=\text{lim}_{t\rightarrow 0^{+}} \frac{1}{t}(T(t)f-f) \quad \text{for} \quad f \in D(A) \]
 
 
Since the generator defined above is a generator of a strongly continuous semi group $T(t)_{t\geq 0}$ we have that for every $f\in X$ the orbit map 
 \[x:t \longrightarrow x(t)=T(t)f\]
 is the unique mild solution of the abstract Cauchy equation (\ref{abstractcauchy2}) 
 
 
 
 
%\subsection{The Finite Difference Method}

\section{Outline Of The Thesis}